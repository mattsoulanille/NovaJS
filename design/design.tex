\documentclass{article}
\usepackage{xspace, hyperref}
\newcommand{\gameData}{\emph{gameData}\xspace}

\title{NovaJS Design}



\begin{document}
\maketitle

\section{Purpose of this Document}
This design document aims to specify every design choice in the NovaJS project and define all interfaces between pieces of the project.

\section{Accessing Game Data}

There is a universal interface, used on both the client and the server, for accessing game data.
\subsection{Game Data Interface}

All game data is accessed from a single object hereafter referred to as \gameData. \gameData has the following data types as fields: \textbf{outfits}, \textbf{picts}, \textbf{planets}, \textbf{spriteSheets}, \textbf{systems}, and \textbf{weapons}. Each data type field has a \textbf{.get} method which takes a string ID of a resource in the field and returns a promise that resolves to the resource corresponding to the ID.\\

The format of each resource is described below. Resources do not have circular references and can be sent over SocketIO.
\subsubsection{Common Fields}
Every resource has the following fields:
\begin{itemize}
\item{\textbf{id}}: The unique global ID of the resource. Same as the ID used to \textbf{.get} it.
\item{\textbf{name}}: The name corresponding to the resource. Only used for player interaction.

\end{itemize}

\subsubsection{Outfit}
An \textbf{outfit} is anything that can be attached to a ship. It has the following fields in addition to the default ones:
\begin{itemize}
\item{\textbf{functions}}: An object whose keys are the general functions an outfit performs and whose values are parameters for those functions. See \href{outfit-functions}{3} for details.
\item{\textbf{weapon}}: An object specifying the weapon (if any) that the outfit has. Specifies the global ID under \textbf{id} and the quantity of the corresponding weapon under \textbf{count}.
\item{\textbf{pictID}}: The global ID of the picture to use in the outfitter.
\item{\textbf{desc}}: The text description of the outfit.
\item{\textbf{mass}}: The mass of the outfit (see \href{outfitter}{Outfitter}).
\item{\textbf{price}}: The price of the outfit (see \href{outfitter}{Outfitter}).
\item{\textbf{displayWeight}}: Determines the order in which the outfit is shown in the \href{outfitter}{Outfitter}.
\item{\textbf{max}}: The maximum number that can be on a single ship.
\end{itemize}

\subsubsection{Pict}
A \textbf{pict} stores a single picture. It has a field \textbf{png} that is a buffer containing the data of the picture in PNG format.

\subsubsection{Planet}
A \textbf{planet} is a stationary stellar object. Space stations fall under this category. A \textbf{planet} has the following fields in addition to the default ones:
\begin{itemize}
\item{\textbf{type}}: A field containing the value \textbf{``planet''} specifying that this object is a planet (maybe change this).
\item{\textbf{position}}: A list specifying the $[x,y]$ position of the planet.
\item{\textbf{landingPictID}}: The global ID of the picture to show when a player lands.
\item{\textbf{landingDesc}}: The text description to show when the player lands.
\item{\textbf{animation}}: An object specifying the spritesheet to use for the planet (and how to use it? Revise this)
\end{itemize}

\subsubsection{SpriteSheetData (TODO)}
A \textbf{spriteSheetData} resource encodes data about the contents of a \textbf{spriteSheet} including frame boundaries and collision convex hulls for each image.

\subsubsection{SpriteSheet}
A \textbf{spriteSheet} resource is identical to a \textbf{pict} resource, however, it exists within a different id scope (\textbf{spriteSheets} instead of \textbf{picts}).

\subsubsection{Systems}
A \textbf{system} resource describes the contents of a system. It includes the following fields:
\begin{itemize}
\item{\textbf{position}}: The $[x,y]$ position of the system on the star map.
\item{\textbf{links}}: A list of the global IDs of \textbf{systems} that this system has a link to.
\item{\textbf{planets}} A list of the global IDs of \textbf{planets} contained in the system.
\end{itemize}





% \begin{itemize}
% \item{outfits}: 
% \item{picts}
% \item{planets}
% \item{spriteSheets}
% \item{systems}
% \item{weapons}

  




\section{Outfit Functions} \label{outfit-functions}

\section{Outfitter} \label{outfitter}


\end{document}
